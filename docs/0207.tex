\documentclass[aspectratio=169]{beamer}

\usepackage{graphics}

\title[Returns to Buying Winners and Selling Losers]{Returns to Buying Winners and Selling Losers}
\subtitle{Based on Jegadeesh \& Titman (1993) JF }
\author[Mahdi Mir]{Mahdi Mir}
\institute{TeIAS}
\date[Apr 2021]{Apr 2021}

\usetheme{CambridgeUS}


\begin{document}

\maketitle

\section{Overview}

\begin{frame}
	\begin{center}
		\resizebox{8cm}{!}{
			\begin{tabular}{rlcccccccccc}
				      &          & \multicolumn{5}{c}{Panel A} & \multicolumn{5}{c}{Panel B}                                                                \\ \cline{3-7}\cline{8-12}
				      & J        & K=                          & 3                           & 6      & 9      & 12     & K= & 3 & 6      & 9      & 12     \\ \hline
				3     & Sell     &                             & -0.022                      & -0.018 & -0.016 & -0.014 &    &   & -0.005 & -0.005 & -0.005 \\
				[1ex] &          &                             & (5.24)                      & (5.07) & (5.06) & (5.07) &    &   &        &        &        \\
				3     & Buy      &                             & 0.0245                      & 0.0190 & 0.0176 & 0.0162 &    &   &        &        &        \\[1ex]
				      &          &                             & (5.89)                      & (5.55) & (5.94) & (6.00) &    &   &        &        &        \\
				3     & Buy-sell &                             & 0.0006                      & 0.0008 & 0.0013 & 0.0013 &    &   &        &        &        \\[1ex]
				      &          &                             & (0.40)                      & (0.79) & (1.74) & (2.06) &    &   &        &        &        \\
				6     & Sell     &                             & -0.020                      & -0.014 & -0.013 & -0.012 &    &   &        &        &        \\[1ex]
				      &          &                             & (5.39)                      & (4.90) & (4.82) & (5.10) &    &   &        &        &        \\
				6     & Buy      &                             & 0.0200                      & 0.0165 & 0.0146 & 0.0129 &    &   &        &        &        \\[1ex]
				      &          &                             & (5.25)                      & (5.80) & (5.82) & (5.60) &    &   &        &        &        \\
				6     & Buy-sell &                             & -0.000                      & 0.0011 & 0.0012 & 0.0003 &    &   &        &        &        \\[1ex]
				      &          &                             & (0.18)                      & (1.16) & (1.54) & (0.65) &    &   &        &        &        \\
				9     & Sell     &                             & -0.020                      & -0.014 & -0.012 & -0.011 &    &   &        &        &        \\[1ex]
				      &          &                             & (5.43)                      & (4.85) & (4.66) & (4.85) &    &   &        &        &        \\
				9     & Buy      &                             & 0.0209                      & 0.0153 & 0.0129 & 0.0111 &    &   &        &        &        \\[1ex]
				      &          &                             & (5.91)                      & (5.89) & (5.74) & (5.45) &    &   &        &        &        \\
				9     & Buy-sell &                             & 3.0158                      & 0.0004 & 0.0005 & -0.000 &    &   &        &        &        \\[1ex]
				      &          &                             & (0.01)                      & (0.40) & (0.64) & (0.40) &    &   &        &        &        \\
				12    & Sell     &                             & -0.017                      & -0.013 & -0.011 & -0.010 &    &   &        &        &        \\[1ex]
				      &          &                             & (4.70)                      & (4.73) & (4.61) & (4.78) &    &   &        &        &        \\
				12    & Buy      &                             & 0.0175                      & 0.0123 & 0.0098 & 0.0087 &    &   &        &        &        \\[1ex]
				      &          &                             & (5.29)                      & (5.07) & (4.82) & (4.82) &    &   &        &        &        \\
				12    & Buy-sell &                             & 8.3600                      & -0.000 & -0.000 & -0.000 &    &   &        &        &        \\[1ex]
				      &          &                             & (0.05)                      & (0.40) & (0.80) & (1.26) &    &   &        &        &
			\end{tabular}}
	\end{center}
\end{frame}


\begin{frame}{Trading Strategies}
	\begin{itemize}
		\item The strategies we consider select stocks based on their returns over the past 1, 2, 3, or 4 quarters. We also consider holding periods that vary from 1 to 4 quarters. This gives a total of 16 strategies.
		\item Specifically, a strategy that selects stocks on the basis of returns over the past J months and holds them for K months (we will refer to this as a J-month/K-month strategy)
		\item At the beginning of each month t the securities are ranked in ascending order based on their returns in the past J months. Based on these rankings, ten decile portfolios are formed that equally weigh the stocks contained in the top decile, the second decile, and so on.
		\item The top decile portfolio is called the " losers" decile and the bottom decile. is called the " winners" decile.
	\end{itemize}
\end{frame}

\begin{frame}{Trading Strategies (con't)}
	\begin{itemize}
		\item In each month t, the strategy buys the winner portfolio and sells the loser portfolio, holding this position for K months.
		\item In addition, the strategy closes out the position initiated in month t - K.
	\end{itemize}
\end{frame}


\section{Data'self Summary}


\begin{frame}{Summary Stats. of the Data}
	\begin{itemize}
		\item Start Date: 1380-01
		\item End Date: 1399-09
		\item The data includes daily open, last, high, low, close prices and volume for each stock, in all days a stock is traded on TSE.
		\item Prices are adjusted.
	\end{itemize}
\end{frame}




\begin{frame}{Monthly Return of Equally Portfolio of All Securities in the Market}
	\begin{itemize}
		\item What if at the start of each Jalali month we buy a portfolio of equal weight of the whole market, and holding it just until next month?
		\item Monthly returns for the latter portfolio are drawn in the next slide.
		\item Yearly returns which are derived from monthly returns are depicted in the next two slide.
	\end{itemize}
\end{frame}




\begin{frame}{Monthly Return of Equally Portfolio of All Securities in the Market}
	\begin{center}
		\begin{tabular}{l r r}
			                       & Monthly     & Annually   \\ \cline{2-3}
			Number                 & 236         & 20         \\
			Positive Return Number & 164         & 17         \\
			Negative Return Number & 71          & 3          \\ \hline
			Returns                &             &            \\ \hline
			Mean                   & 3.43\%      & 57.73\%    \\
			Var.                   & 0.0050      & 0.6598     \\
			STD                    & 0.0711      & 0.8123     \\
			Median                 & 1.98\%      & 31.06\%    \\
			Max.                   & 43.23\%     & 369.51\%   \\
			Min.                   & -11.20\%    & -8.69\%    \\
			Cumulative             & 172687.60\% & 76764.42\%
		\end{tabular}
	\end{center}
\end{frame}


\section{Results}
\subsection{Most Significant Results}

\begin{frame}{Most Significant Results}
	\begin{itemize}
		\item First 100 most Significant results in the sense of being different from zero, have all Positive returns.
		\item There are 1651 Strategies with significance level more then 1\%.
		\item 850 of them have positive returns.
		\item Most significant positive returns are tabulated in the next slide.
	\end{itemize}
\end{frame}



\end{document}
